\documentclass[12pt,a4paper]{article}
\usepackage{geometry}
%\setlength{\parindent}{0pt}
\usepackage{amssymb}
\usepackage{amsmath}
\usepackage{amsthm}
\usepackage{sectsty}
\usepackage{graphicx}
\usepackage[square,numbers]{natbib}
\usepackage[utf8]{inputenc}
\usepackage[english]{babel}
\usepackage{caption}
\usepackage{subcaption}
\usepackage[hidelinks]{hyperref}
\usepackage[table]{xcolor}

\geometry{top=2cm}
\geometry{bottom=2cm}
\geometry{left=2cm}
\geometry{right=2cm}
\geometry{bindingoffset=0cm}

\bibliographystyle{abbrvnat}
%\subsectionfont{\normalfont\itshape}

%\theoremstyle{definition}
\newtheorem{definition}{Definition}[section]
\newtheorem{theorem}{Theorem}[section]

\DeclareMathOperator{\Var}{Var}
\DeclareMathOperator{\VaR}{VaR_\alpha}

\title{Probabilistic Multivariate Time Series Forecasting using Deep Generative Models}
\date{\today}

\author{
    \small Vitaliy Pozdnyakov \\
    \small Faculty of Computer Science \\
    \small \textit{National Research University Higher School of Economics} \\
    \small Moscow, Russia\\
    }

\begin{document}

%\maketitle

\abstract{Coming soon.}

\newpage

\tableofcontents

\newpage

\section{Introduction}

Time series forecasting is a crucial task for decision making in many areas. For example, an accurate demand forecast helps retailers to maintain required in-stock rate of products, improve supply chain operations and reduce delivery time. Deciding whether to build a new power plant requires electricity demand forecasts, scheduling staff in a call centre requires forecasts of call volumes. 

We are trying to forecasts an unknown value and thus we can consider it as a random variable which means that there are many possible future outcomes. There can also be observed the butterfly effect — small deviations in the beginning lead to drastical changes in the future and therefore the relatively near future is more predictable than the distant one. For example, tomorrow sales can be forecasted with high confidence, while next year sales are much more vague. The such uncertanty of a forecast can be represented as a prediction interval of possible values that the random variable could take with relatively high probability. For example, 95\% prediction interval should contain a range of values that will be observed with probablity 0.95. From a statistical point of view, there are possible two types of forecasts: point and probabilistic \cite{fpp3}. A point forecast is a single predicted value, for example the average of possible future values, while probabilistic forecast is a prediction interval or a set of prediction intervals for different confidence levels, say 50\%, 90\%, 95\%, 99\%. The bigger confidence level, the larger prediction interval and vice versa, so the 0\% prediction interval is a single value that actually is the median which means that a half of possible values lie above the interval and a half below it. Note that 0\% prediction interval can also be considered as a point forecast. Forecasting prediction intervals can also be formulated as quantile regression where lower and upper bounds of interval are quantiles, for example 95\% prediction interval can be represented as quantiles 0.025 and 0.975.

In a rapidly changing environment, probabilistic forecasts are becoming increasingly important as they help to understand an uncertanty and underlying risks. For example, selection of target quantiles based on historical sales and stock-out costs of products makes forecasts useful for inventary planning. The fact that sales of modern smartphones can have significant peaks can require a quantile as high as 0.9 to protect againts the risk of the lack of stock. Probabilistic forecast are used in any situations where robust decisions should be made against uncertanty, such as budgeting decisions, trading and hedging strategies. For example, commodity trading companies can decide to optimize a portfolio to reduce risks if a range of supposed losses/profit is too wide. Novel risks involved by COVID-19 pandemic and climate changes clearly show that uncertanty assessment play a key role in the decision making. Another rationale for probabilistic forecasting is that it can determine how much to trust the predictions for underlying problems, as well as distinguish between regions with high and low uncertanty. For example, seasonal sales can be accurately forecasted for periods with high demand, while other periods can have wide range of possible sales.

In many cases, one-dimensional time series are statistically dependent on each other, and this dependency can also be taken into account to improve the accuracy of forecasts \cite{normflow2021}. For example, forecasting total sales of two negatively correlated products involves the fact that simultaneously increasing sales of both products is unrealistic, while independent forecasts cannot account this property. The effect of interacting items is known as the cannibalization effect and it is of great importance in the retail business. In addition, in dealing with economic variables, the value is often not only related to its historical movements, but also to historical movements of other variables. For example, interest rate, income, and investment expenditures may have a large impact on the forecasting household consumption expenditures. Thus, it makes sense to use all available variables using a multivariate forecast instead of an individual ones.

Despite the fact that classical statistical methods of time series forecasting are often used by forecasting practitioners \cite{rnn2019}, recently more and more scientific papers on the use of neural networks for time series forecasting have appeared. Most of them are based on recurrent neural networks, attention-based models, and temporal convolutional networks \cite{tsdeeplearning2021}. In addition, recent public competitions in time series forecasting, such as M4 in 2018 \cite{MAKRIDAKIS202054} and M5 in 2020 \cite{m52020}, have shown that the best quality is achieved by gradient boosting models and neural networks \cite{m52020}. Previously, the use of neural networks was complicated by the fact that they are much more complex than classical statistical models and therefore require more historical data for training, in addition, they require specialists to understand the work and tuning of neural networks. However, now we live in the era of Big Data, companies collect a huge amount of data from day to day, which contains important information about the patterns of their business. In conditions where high-precision forecasting requires the use of all available information, complex models such as neural networks gain a great advantage.

Neural networks have already become popular in many data analysis tasks related to image, video, audio, and text processing. In addition, generative models of neural networks (deep generative models) are actively used in such tasks \cite{introductiondgm2021}. A distinctive feature of such models is that they are able to generate new observations based on previously seen ones. For example, generative adversarial networks (GANs) proposed in \cite{goodfellow2014} are actively used to generate new images \cite{choi2018stargan} and sound \cite{oord2016wavenet}. However, the use of such models for generating time series is only beginning to develop. Generative models can be used not only for generating time series and then obtaining a probabilistic forecast \cite{koochali2020like}, but also for backtesting and improvement the robustness of business strategies \cite{lezmi2020improving}. For example, to check the quality of a trading strategy, the time series is divided into two periods: "in-sample" (train period) and "out-of-sample" (test period). The purpose of this separation is to use the best parameters found on the "in-sample" period and then test the quality of such a strategy on the "out-of-sample" period. This approach has two drawbacks. First, it reduces the number of observations available for training the model, in addition to removing the most relevant observations from the train period, which may contain important information about the trend, seasonality changes and so on. Secondly, it is still based on only one validation split, since we observe only one realization of an unknown random process. To partially solve the second drawback in \cite{fpp3}, a cross-validation method for time series was proposed, which is based on the sequential division of the time series into pairs of periods "in-sample" and "out-of-sample". Generative models offer a new approach to solving this problem — to generate a time series whose characteristics will correspond to the real time series and then get a set of pairs of periods "in-sample" and "out-of-sample". The peculiarity of deep generative models is that they are able to take into account the characteristics of real financial time series: nonlinear autocorrelations, heavy tails, heteroskedasticity, regime change, and non-stationary properties \cite{lezmi2020improving}. For example, the Quant GAN model, as shown in \cite{quantgan2020}, is able to model such characteristics of financial time series.

Several deep generative models for time series forecasting have been proposed in recent years. All of them showed excellent quality and superiority over their competitors. At the same time, different models were considered as the baselines, different datasets, and different quality metrics were used in these works. Some of the models were tested on univariate forecasting, while others were tested on multivariate forecasting. The objectives of this work are:
\begin{itemize}
    \item Comparison of such models with each other on the same datasets and quality metrics.
    \item Presentation of the taxonomy of such models and possible directions for their development and modifications.
    \item Consideration of the impact on the quality of these models of classical time series preprocessing steps: Box-Cox transformation and differencing.
\end{itemize}
DESCRIBE THE SECTIONS

\section{Literature reviews}

\section{Background}

\subsection{Probabilistic modelling}

We start from the point (deterministic) forecasting model. Let $y_t \in \mathbb R$ be a single value at the time moment $t$ and $y_1, y_2, \dots, y_T$ be known historical values. For simplicity, let us introduce the notation $y_{1:T} = y_1, y_2, \dots, y_T$. Then the $h$-step-ahead forecasting model can be defined as 
$$\hat{y}_{T+h} = f(y_T, y_{T-1}, \dots)$$
where $h$ is a predictable horizon. For example, $h=1$ for daily time series data means forecasting tomorrow's values. Multivariate forecasting model can be naturally generalized as
$$\hat{\mathbf y}_{T+h} = f(\mathbf y_T, \mathbf y_{T-1}, \dots)$$
where $\mathbf y \in \mathbb R^K$ and $i \in \{1, \dots, K\}$ is an identificator of an individual component of the time series \cite{mts2007}. For example, $y_{1,t}$ can represents sales of the item 1 and $y_{2,t}$ is sales of the item 2. 

Generalizing the deterministic model, we can introduce the probabilistic forecasting model. We denote the triple ($\Omega$, $\mathcal F$, $P$) as a probability space, where $\Omega$ is the set of elementary events, $\mathcal F$ is a sigma-algebra of these events, and $P$ is the probability measure defined on $\mathcal F$. The $K$-dimensional random vector $\mathbf y$ is a function defined on $\Omega$ such that for any real $K$-dimensional number $c$ there exists an event $A_c$ and
$$A_c = \{ \omega \in \Omega | y_1(w)\leq c_1, \dots, y_K(w) \leq c_K\} \in \mathcal F$$
Thus, the joint distribution function $F: \mathbb R^K \to[0, 1]$ is defined as $F(c)=P(A_c)$. Let us denote $p_\mathbf z$ as a set of indexes described a time axis, say positive integers $Z = \{1, 2, \dots\}$. Then we define a $K$-dimensional vector stochastic process as a function 
$$\mathbf y: Z \times \Omega \to \mathbb R^K$$
For any fixed $t \in Z$, the proccess $\mathbf y_t$ is a $K$-dimensional random vector. Thereby, the any fixed $\omega$ describes some realization of the stochastic process $\mathbf y_1(\omega), \mathbf y_2(\omega), \dots$. The underlying stochastic process is often called the data generating process, or generative model \cite{mts2007}.

In terms of probabilistic forecasting, we want to model a conditional joint distribution function of forecasted values with given historical observations
$$F(\mathbf y_{T+h} | \mathbf y_{T}, \mathbf y_{T-1}, \dots)$$
and then calculate point forecasts or prediction intervals from this distribution. Often we work with absolutely continuous random vectors and then we model a joint probability density function
$$p(\mathbf y_{T+h} | \mathbf y_{T}, \mathbf y_{T-1}, \dots) = \frac{\partial^K F(\mathbf y_{T+h} | \mathbf y_{T}, \mathbf y_{T-1}, \dots)}{\partial y_{1, T+h}\dots\partial y_{K, T+h}}$$
Thereby, a point forecas can be obtained as the conditional expectation
$$\hat{\mathbf y}_{T+h} = \mathbb E[\mathbf y_{T+h} | \mathbf y_{T}, \mathbf y_{T-1}, \dots]$$
while, for example, the 0.95 quantile of $i$-th time series is obtained from a marginal cumulative distribution function as
$$\hat{y}_{i, T+h} = \{ y_{i, T+h} | F(y_{i, T+h} | y_{i, T}, y_{i, T-1}, \dots) = 0.95 \}$$
Note the fact that marginalization makes the prediction intervals of the different components independent, and as a result, the forecast loses information about how realistic future movements of certain components will be.

\subsection{Vector autoregressive model}

The Vector Autoregressive (VAR) model is a classical statistical method for the multivariate forecasting \cite{mts2007} and it is the generalization of univariate autoregressive (AR) model. The VAR model of the order $p$ is denoted as VAR($p$) and it is defined via a system of linear equation
$$\mathbf y_{t} = \nu + A_1 \mathbf y_{t-1} + A_2 \mathbf y_{t-2} + \dots + A_p \mathbf y_{t-p} + \varepsilon_t$$
where $A_i$ are fixed ($K\times K$) coefficient matrices, $\nu$ is a fixed ($K \times 1$) intercept vector that is used in case of a non-zero mean of some individual time series, and $\varepsilon$ is a $K$-dimensional white noise or innovation process that holds the properties $\mathbb E [\varepsilon_t] = 0$, $\mathbb E [\varepsilon_t \varepsilon_t^\top] = \Sigma_\varepsilon$ and $\mathbb E [\varepsilon_t \varepsilon_s^\top] = 0$ for any $t\neq s$. Note that the noise vector is distributed as $\mathcal N(0, \Sigma_\varepsilon)$ and can be correlated among individual series, but $\varepsilon_t$ and $\varepsilon_s$ are independent if $t \neq s$.

The forecasting via VAR model is performed in a recursive manner. $p$ historical vectors $\mathbf y_{t-1}, \dots, \mathbf y_{t-p}$ are fed into VAR model and probabilistic forecasting of individual time series is calculated as
$$y_{i, t} \sim \mathcal N([\nu + A_1 \mathbf y_{t-1} + A_2 \mathbf y_{t-2} + \dots + A_p \mathbf y_{t-p}]_i, [\Sigma_\varepsilon]_{i,i})$$
where $[\cdot]_i$ is an i-th element of the vector and $[\cdot]_{i,i}$ is the element on an i-th row and i-th column of the matrix. For example, if $[\nu + A_1 \mathbf y_{t-1} + A_2 \mathbf y_{t-2} + \dots + A_p \mathbf y_{t-p}]_i = m$ and $[\Sigma_\varepsilon]_{i,i} = \sigma^2$ then the 95\% prediction interval be $m \pm 1.96 \sigma$. Next, the expected  vector $\hat {\mathbf y}_t = \mathbb E [\mathbf y_t]$ are fed into VAR model to obtain probabilistic forecast for $\mathbf y_{t+1}$.

The fitting of the parameters $\nu, A_1, \dots, A_p, \Sigma_\varepsilon$ can be performed by maximum likelihood (ML) estimation or least sqares (LS) estimation \cite{mts2007}.

The selection of the hyperparameter $p$ is performed by minimizing the Bayesian Information Criterion (BIC) that defined as
$$\text{BIC} = m \ln (n) + 2 \ln (\hat L)$$
where $m$ is the number of parameters estimated by the model that equivalent to $pK^2 + K$ in case of the VAR($p$) model, $n$ is the number of observations and $\hat L$ is the likelohood of the model.

\subsubsection{Transformations for stationarity}

There are three assumptions under the VAR model. First, we assume that the stochastic process is stationary, that is $\mathbb E (\mathbf y_t) = \mu$ for all $t$ and $\mathbb E [(\mathbf y_t - \mu)(\mathbf y_{t-h}-\mu)^\top] = \Gamma(h) = \Gamma(-h)^\top$ for all $t$ and $h$. Second, the determenistic part of components of stochastic proccess linearly depend on each other. Third, the errors of forecasts are described by the white noise.

The most time series are non-stationary in practice, but they can be transformed into stationary using the differencing operator in some cases
$$\Delta^d y_t = (1 - L)^d y_t$$
where $L$ is the lag operator $Ly_t = y_{t-1}$. The times series that can be transformed into stationary by $d$-differencing is called integrated of order $d$. To check that the time series is integrated of order $d$ is performed unit root tests, for example Kwiatkowski-Phillips-Schmidt-Shin (KPSS) test \cite{fpp3}. To stabilize the variance, there is common practice to use Box-Cox transformation
$$y_t^* = \begin{cases}
    \ln y_t & \text{if $\lambda = 0$} \\ 
    \frac{y_t^\lambda - 1}{\lambda} & \text{otherwise}
\end{cases}$$
where the parameter $\lambda \in \mathbb R$ can be selected by Guerro estimation \cite{guerrero93}.

\subsection{Deep generative models} \label{deepgenmodels}

The main goal of generative modeling is to obtain a representation of the intractable distribution $p_\mathbf y$ defined on $\mathbb R^K$. As a rule, $K$ is quite large and the distribution is quite complex. As a training set, we have independent identically distributed observations from the distribution $p_\mathbf y$. Our goal is to find a generator that can map observations from a simple prior distribution $p_\mathbf z$ defined on $\mathbb R^q$ to the distribution $p_\mathbf y$
$$G: \mathbb R^q \to \mathbb R^K$$
In other words, the generator translates samples (random noise) from a simple distribution $\mathbf z \sim p_Z$ to a distribution $\mathbf y \sim p_\mathbf y$ such that $G(\mathbf z) \approx \mathbf y$. As a common practice, $K$ i.i.d. Gaussian random variables $\mathcal N(0, 1)$ are selected as the prior distribution, that is $p_{\mathbf z} = \mathcal N(0, I_K)$.

In terms of the time series forecasting, we want to model the conditional distribution $p_\mathbf y(\mathbf y_{T+h} | \mathbf y_{T}, \mathbf y_{T-1}, \dots)$ and it can be approximated as
\begin{equation}\label{eq:gen}
\begin{aligned}
p_\mathbf y(\mathbf y_{T+h} | \mathbf y_{T}, \mathbf y_{T-1}, \dots) &= \int p_\mathbf y(\mathbf y_{T+h}, \mathbf z| \mathbf y_{T}, \mathbf y_{T-1}, \dots)\mathrm{d}\mathbf z\\ 
&= \int p_\mathbf y(\mathbf y_{T+h} | \mathbf z, \mathbf y_{T}, \mathbf y_{T-1}, \dots)p(\mathbf z) \mathrm{d}\mathbf z\\ 
&= \mathbb E_\mathbf z[\mathbf y_{T+h} | \mathbf z, \mathbf y_{T}, \mathbf y_{T-1}, \dots]\\ 
&\approx G(\mathbf z, \mathbf y_{T}, \mathbf y_{T-1}, \dots)        
\end{aligned}
\end{equation}
Thus, the generator should be able to model the conditional distribution with the given historical observations so that $G(\mathbf z, \mathbf y_{T}, \mathbf y_{T-1}, \dots) \approx \mathbf y_{T+h}$ where $\mathbf z \sim p_Z$ and $\mathbf y_{T+h} \sim p_\mathbf y(\mathbf y_{T+h} | \mathbf y_{T}, \mathbf y_{T-1}, \dots)$.

For many datasets with high dimensionality and complex dependencies between components, it is becoming common practice to use deep neural networks as universal approximators \cite{introductiondgm2021}. Such neural networks can be denoted as $G_\theta$, where $\theta$ means the trainable parameters of the neural network. The training of the generator is performed by updating trainable parameters $\theta$ by gradint descend with respect to the loss function $\mathcal L$ that somehow measures the dissimilarity between the generated distribution and the observed one.

\subsubsection{Direct parametric approach}

One of the most simple methods of modeling a distribution of time series using neural networks is to train the parameters of some predetermined distribution \cite{tsdeeplearning2021}. For example, the multivariate Gaussian distribution is often used so that
$$\mathbf y_{T+h} \sim \mathcal N(\mu(\mathbf y_{T}, \mathbf y_{T-1}, \dots), \Sigma(\mathbf y_{T}, \mathbf y_{T-1}, \dots))$$
where $\mu$ and $\Sigma$ are outputs of a neural network. The mean vector $\mu$ has a dimension of $K \times 1$ while the covariance matrix $\Sigma$ has a dimension of $K \times K$, so the number of estimated parameters increases quadratically with the growth of the time series components, which makes learning very difficult. To solve this problem, a diagonal covariance matrix is used, $\Sigma_{ij} = 0$ for all $i \neq j$. In this case, the matrix is described only by diagonal elements that represented by a single vector $K \times 1$. Thus, the output of the network is $2K$-dimensional. To ensure that the standard deviations take only positive values, we can add softplus function before the network outputs $\Sigma$.

Fitting the network parameters is performed by the maximum likelihood estimation (or that equivalent, minimization of the negative log-likelihood). For example, for the multivariate Gaussian distribution with diagonal covariance matrix, we can consider components as independent variables and then for any individual target $p_\mathbf y$, we have a pair of mean $\mu_i$ and standard deviation $\sigma_i$. In this case, the negitive log-likelihood is calculated as
$$\begin{aligned}
L_{G} &=-\ln \ell(\mu, \sigma \mid y) \\
&=-\ln \left(\left(2 \pi \sigma^{2}\right)^{-1 / 2} \exp \left[-(y-\mu)^{2} /\left(2 \sigma^{2}\right)\right]\right) \\
&=\frac{1}{2} \ln (2 \pi)+\ln (\sigma)+\frac{(y-\mu)^{2}}{2 \sigma^{2}}
\end{aligned}$$

In this limit case, the generator is the identity function $G(\mathbf z, \mathbf y_{T}, \mathbf y_{T-1}, \dots) = \mathbf z$, because the forecasts are already described by tractable simple distribution. We can easily generate future observations from the multivariate Gaussian distribution to obtain a probabilistic forecast.

\subsubsection{Generative adversarial network}

Generative adversarial networks (GANs) are a distribution modeling strategy that uses two neural networks: the generator $G$ and the discriminator $D$. The generator $G$ performs the task of translating samples from a relatively simple distribution of $p_\mathbf z$ to a rich, complex, multidimensional distribution of a stochastic process $p_\mathbf y$. The task of discriminator $D$ is to distinguish the true data from the generated data. The generated and true data in the GAN context are called fake and real data. Strictly speaking, the discriminator's task is to minimize the binary classification error from input data that contains both real and fake data. On the other hand, the task of the generator is to generate such fake data so that it is as difficult as possible for the discriminator to distinguish them from the real ones. The generator and discriminator are trained simultaneously, so they compete until the generator is able to generate indistinguishable data from the real data.

From the mathematical point of view, a generator is a function that translates vectors of one space into a vector of another space $G: \mathbb R^q \to \mathbb R^K$, as described in the section \ref{deepgenmodels}. Let say that the generator model the distribution $p_\mathbf{\hat y}$ that we want to as close as possible to the real distribution $p_\mathbf y$. The discriminator is a function that evaluates how likely the observation is to be real, $D: \mathbb R^K \to [0, 1]$, where 0 means fake data and 1 means real data. Thus, 0.5 means that for the discriminator, such data looks equally real and fake. As previously mentioned, the generator and discriminator are trained simultaneously, that is, they play a two-player minimax game with the value function $V$
$$\begin{aligned}
\min _{G} \max _{D} V(D, G) =\mathbb{E}_\mathbf{y}[\ln D(\mathbf{y})]+ \mathbb{E}_\mathbf{z}[\ln (1-D(G(\mathbf{z})))]
\end{aligned}$$
The training of the generator and discriminator takes place in steps, so the value function is decomposed into two functions: discriminator's loss function 
$$L_D = -\mathbb{E}_\mathbf{y}[\ln D(\mathbf{y})] - \mathbb{E}_{\mathbf{z}}[\ln (1-D(G(\mathbf{z})))]$$
and generator's loss function
$$L_G = -\mathbb{E}_{\mathbf{z}}[\ln (D(G(\mathbf{z})))]$$
according to which the weights of the neural networks are updated step-by-step. As a rule, $l>1$ steps of the discriminator optimization are performed, and then the single step of the generator optimization is made.  As was shown in \cite{goodfellow2014}, the value function with the optimal discriminator $D^*$ is 
$$V(D^*, G)=-\ln (4)+2 \cdot \text{JSD}(p_\mathbf{y} \| p_\mathbf{\hat y} )$$
where JSD is the Jensen–Shannon divergence between the model’s distribution and the data generating process. Thus, the generator learns to minimize the JSD and then in the case of identical distributions $p_\mathbf{y}$ and $p_\mathbf{\hat y}$, the value function is in the global optimum $V^* = - \ln (4)$.

GANs for probabilistic forecasting has been proposed in \cite{koshiyama2019generative} and \cite{koochali2020like}. Here we want to model the conditional distribution $p_\mathbf y(\mathbf y_{T+h}|\mathbf{y}_{T}, \mathbf{y}_{T-1}, \dots)$ via the generator $G$($\mathbf z$, $\mathbf{y}_{T}$, $\mathbf{y}_{T-1}$, $\dots$), and then the discriminator should classify the sequence $\mathbf y_{T+h}, \mathbf{y}_{T}, \mathbf{y}_{T-1}, \dots$, where $\mathbf{y}_{T}, \mathbf{y}_{T-1}, \dots$ are historical observations, and $\mathbf y_{T+h}$ is real or fake. Thereby, the full conditional value function $V$ is 
$$
\begin{aligned}
\min _{G} \max _{D} V(D, G) =\mathbb{E}_\mathbf{y}[\ln D(\mathbf y_{T+h} | \mathbf{y}_{T}, \mathbf{y}_{T-1}, \dots)]\\
+ \mathbb{E}_\mathbf{z}[\ln (1-D(G(\mathbf z | \mathbf{y}_{T}, \mathbf{y}_{T-1}, \dots) | \mathbf{y}_{T}, \mathbf{y}_{T-1}, \dots))]    
\end{aligned}
$$
where $G(\mathbf z | \mathbf{y}_{T}, \mathbf{y}_{T-1}, \dots)$ is a probabilistic form of the generator \eqref{eq:gen} in the sence that $\mathbf z$ is the random variable, but not a sample from the prior distribution.

\subsubsection{Normalizing flow}

The Normalizing flow \cite{normflow2021} is the generator of the form $G: \mathbb R^q \to \mathbb R^K$ where $q=K$ and consists of a sequence of invertible functions. Each function $f$ makes the transformation of initial density function as follows
$$p_{\mathbf{y}}(\mathbf{y})=p_{\mathbf{z}}(\mathbf{z})\left|\operatorname{det}\left(\frac{\partial f(\mathbf{y})}{\partial \mathbf{y}}\right)\right|$$
where $\partial f(\mathbf y)/\partial \mathbf y$ is the Jacobian of $f$ at $\mathbf y$. The main property of normalizing flows is they are designed so that the computing the Jacobian determinant takes no more than  $O(K)$ time and makes it easy to invert $f^{-1}(\mathbf z) = \mathbf y$.

There are a few options to define such functions. For example, Real NVP \cite{normflow2021} model is constructed on coupling layers that keep some inputs unchanged and transforms other inputs with respect to unchanged ones
$$
\begin{cases}
    \mathbf w^{1:k} = \mathbf y^{1:k} \\
    \mathbf w^{k+1:K} = \mathbf y^{k+1:K} \odot \exp(s(\mathbf y^{1:k})) + t(\mathbf y^{1:k})
\end{cases}
$$
where a superscript is the index of an input vector, $\odot$ is the Hadamard product, $s$ is a scaling, $t$ is a translation that are neural networks that map $\mathbb R^k \to \mathbb R^{K-k}$. To model the target distribution, $M$ coupling layers are connected together so that $\mathbf{y} \to \mathbf{w}_1 \to \mathbf{w}_2 \to \dots \to \mathbf{w}_{M-1} \to \mathbf{z}$. Using transormation density formula, the log likelihood of the target distribution can be written as
$$\log p_{\mathbf{y}}(\mathbf{y})=\log p_{\mathbf{z}}(\mathbf{z})+\log |\operatorname{det}(\partial \mathbf{z} / \partial \mathbf{y})|=\log p_{\mathbf{z}}(\mathbf{z})+\sum_{i=1}^{M} \log \left|\operatorname{det}\left(\partial \mathbf{w}_{i} / \partial \mathbf{w}_{i-1}\right)\right|$$
The log determinant of the Jacobian is calculated very simply, given that the Jacobian itself in this case has a block-triangular shape 
$$\log \left|\operatorname{det}\left(\partial \mathbf{w}_{i} / \partial \mathbf{w}_{i-1}\right)\right| = \log \left| \exp \left(\operatorname{sum}\left(s_{i}\left(\mathbf{w}_{i-1}^{1: d}\right)\right) \right) \right|$$
where $\operatorname{sum}$ is overall summation \cite{normflow2021}. Using such log-likelihood function we can define the generator's loss as
$$L_G = -\log p_\mathbf{y}(\mathbf y)$$
For probabilistic forecasting, the conditioning $p_\mathbf{y}(\mathbf{y}_{T+h}| \mathbf{y}_T, \mathbf y_{T-1}, \dots)$ can be performed via concatenation in every coupling layer $s(\operatorname{concat}(\mathbf w_i^{1:k}, \mathbf h_T))$ and $t(\operatorname{concat}(\mathbf w_i^{1:k}, \mathbf h_T))$ where $\mathbf{h}_T$ is some embedding of the historical observations $\mathbf y_{T}, \mathbf y_{T-1}, \dots$ that can be obtained from another neural network such as RNNs or self-attentions \cite{normflow2021}.

\subsection{Architectures for time series forecasting}

A neural network consists of interconnected layers that consist of neurons. Each neural network has an "input", "output" and some "hidden" layers. The number of neurons on the input layer determines the number of input variables. The number of neurons on the output layer determines the dimension of the output value. For example, to solve the regression problem $y = f (x_1, x_2, x_3)$, a neural network will need three input neurons and one output. Each layer has a set of trainable parameters that are updated during the training of the neural network. 

The simplest neural layer is the linear layer that has a matrix of trainable parameters $W$ and an intercept $b$. The linear layer performs a matrix multiplication of the form $xW^\top + b$. After each linear layer, an activation function is added that allows the neural network to learn complex nonlinear dependencies between variables. An example of such an activation function would be $\text{ReLU} (x) = \max (0, x)$. A neural network that consists only of linear layers and activation functions between layers is called a multi-layer perceptron (MLP) or feed forward network. 

To train neural networks, the back propagation mechanism is used. The idea is that the training of a neural network takes place step by step. Each step consists of two stages. The first stage is forward: data from the training sample is fed to the input of the neural network, and the output value is compared with the correct answer and the value of the loss function (error) is calculated. for example, the loss function $\text{MSE} = (y - \hat y)^2$ where $y$ is correct answer and $\hat y$ is an output of the network. This loss function can be used for the regression task. The second backward stage is the calculation of the gradient of the loss function for performing gradient descent. Since the neural network can be quite complex and the calculation of the analytical solution is difficult, a step-by-step calculation of the gradient for each layer is performed, starting from the last one, which is then passed to the earlier layers using the chain rule. The training of the neural network is performed by updating trainable parameters $\theta$ by gradint descend so that
$$\theta \leftarrow \theta - \alpha \nabla_\theta L$$
where $\alpha$ is the learning rate and $\mathcal L$ is the network's loss function that somehow measures the dissimilarity between the generated distribution and the observed one.

Different combinations of layers give different neural network architectures. Some of these architectures are popular in time series forecasting problems, but these architectures themselves are only applicable in point forecasting. To get a probabilistic forecast, we need to use some kind of generative model. The exception is quantile regression models, which is essentially a set of point forecasts, but since a set of quantiles cannot define a joint distribution, such models are not considered in this work. Note that the generative models described earlier can be built on different architectures, so any combination of "generative model" + "architecture" determines a separate option for obtaining a probabilistic forecast.

\subsubsection{Neural network autoregression}

Neural network autoregression (NNAR) was described in \cite{fpp3}. Such a neural network is a feed forward network with a single hidden layer. The $\text{NNAR}(p,P,k)_m$ model is defined by 4 parameters, where $p$ is the number of previous observations to be input, $P$ is the number of seasonal observations, $k$ is the number of neurons on the hidden layer, and $m$ is the step for taking seasonal observations. For example, for the $\text{NNAR}(3, 2, 2)_{12}$ model, previous observations $y_{t-1}, y_{t-2}, y_{t-3}$, seasonal previous observations $y_{t-12}, y_{t-24}$ will be fed in input, and a hidden linear layer with two neurons will be used for the model. By default, the output of the neural network is the forecasted next value $y_t$, so for forecasting the series at step $t+h$, it is performed according to the usual autoregressive model — the previously forecasted value is fed in input. However, probabilistic forecasting requires a more flexible architecture in which we can use as many outputs as we need to implement the desired model. Therefore, for example, to implement the GAN model, we can use two NNAR neural networks, in which the generator will have the same number of outputs as the number of discriminator inputs.

\subsubsection{Recurrent neural network}

Recurrent neural networks (RNNs) were proposed for sequence modeling, and they were especially popular in natural language processing (NLP) tasks. The sequence of words in a sentence is modeled by a time series, so time series forecasting has also become a natural use of such models. The main feature of such neural networks is that they have an internal memory that stores a compact representation of past observations. The internal memory is iteratively updated when new observations are received. RNN contains temporal block, each block in contains a few layers. For example, the Elman RNN block is defined as follows
$$
\begin{aligned}
    h_{t} &=\sigma_{h}\left(W_{h_1} x_{t}+W_{h_2} h_{t-1}+b_{h}\right) \\
    y_{t} &=\sigma_{y}\left(W_{y} h_{t}+b_{y}\right)
\end{aligned}
$$
where $h_t$ is a hidden state of the cell, $x_t$ is a input sequence, $y_t$ is the output sequence and $\sigma$ is an activation function. Since the model was originally developed for NLP problems, so there is an input and output sequence here. For example, for a machine translation task, $x_t$ can be a text in Russian, and $y_t$ can be a text in English. When forecasting time series, we work with a whole sequence, so to work with time series, we can use a special version of the modelling, in which $y_t = x_{t+1}$. A feature of recurrent neural networks is also that they do not require the selection of a certain number of previous observations, since the observations are processed sequentially one by one.

Due to some problems in the training (exploding and vanishing gradient) of Elman RNN, several other variants of recurrent networks were eventually developed. For example, the Long short-term memory (LSTM) model uses a special cell block state that stores long-term information transmitted through a sequence of observations. The LSTM block is defined via input gate, output gate, forget gate, hidden state and cell state as follows
$$
\begin{aligned}
\text { input gate: } {i}_{t} &=\sigma\left({W}_{i_{1}} {h}_{t-1}+{W}_{i_{2}} y_{t}+{b}_{i}\right) \\
\text { output gate: } {o}_{t} &=\sigma\left({W}_{o_{1}} {h}_{t-1}+{W}_{o_{2}} y_{t}+{b}_{o}\right) \\
\text { forget gate: } {f}_{t} &=\sigma\left({W}_{f_{1}} {h}_{t-1}+{W}_{f_{2}} y_{t}+{b}_{f}\right)
\end{aligned}
$$
and the states are
$$
\begin{aligned}
\text { hidden state: } h_{t} &=o_{t} \odot \tanh \left(c_{t}\right) \\
\text { cell state: } c_{t} &=f_{t} \odot c_{t-1}+i_{t} \odot \tanh \left(W_{c_{1}} h_{t-1}+W_{c_{2}} y_{t}+b_{c}\right)
\end{aligned}
$$
where $\tanh(\cdot)$ is the tanh activation fucntion. We can obtain the next observation $y_{t+1}$ using a linear layer so that
$$y_{t+1} = W_y h_{t} + b_y$$
\subsubsection{Attention-based model}

The attention mechanism has been proposed to improve long-term dependency learning \cite{bahdanau2016neural}. The transformer model was based entirely on the attention mechanism and achieved the best results in NLP tasks \cite{vaswani2017attention}. Recently, it was shown that the attention mechanism can improve perfomance of time series forecasts with respect to recurrent neural network \cite{tsdeeplearning2021}. The special feature of the attention mechanism is that it aggregates information using dynamic weights, allowing to literally "pay attention" to the right places in the sequence. One modification of the multi-head self-attention mechanism for time series forecasting was proposed in \cite{normflow2021}. The time series $Y = [\mathbf y_{T-t}, \dots, \mathbf y_{T-1}]^\top \in \mathbb R^{t \times K}$ are transformed into $h$ keys, queries and values that computed as
$$
\begin{aligned}
\text{keys: } K_i = Y W_i^K \\
\text{queries: } Q_i = Y W_i^Q \\
\text{values: } V_i = Y W_i^V \\
\end{aligned}
$$
where $W_i^Y \in \mathbb R^{K \times d_k}, W_i^Q \in \mathbb R^{K \times d_k}, W_i^V \in \mathbb R^{K \times d_v}$ are trainable matricies. The attention mechanism produces "heads" of the form
$$
\text{head}_i=\operatorname{Attention}\left(Q_i, K_i, V_i\right)=\operatorname{softmax}\left(\frac{Q_i K_i^{\top}}{\sqrt{d_{k}}} \cdot M\right) V_i
$$
where $M$ is a mask that can filter the rightward observations by defining all upper-triangular element as $-\infty$ so that we cannot pay attention to future time moments. Here the normalization by $d_k$ the dimension of the matricies is performed. The output of the attention layer is
$$O = \text {Concat}\left(\text{head}_{1}, \ldots, \text {head}_{h}\right) W^{O}$$
where $W^O \in \mathbb R^{hd_v \times d_\text{model}}$ and thereby $O \in \mathbb R^{t \times d_\text{model}}$. The attension layer is used in encoder-decoder manner so that we can set $d_\text{model} = K$ to obtain the forecasted time-series in the same dimensionality as the input sequence.

\subsubsection{Temporal convolutional network}

Temporal convolutional network was proposed in \cite{BaiTCN2018}

\section{Experimental setup}

\section{Metrics}
\subsection{Quantile loss}
\subsection{Autocorrelation loss}
\subsection{Correlation loss}

\section{Datasets}

\section{Numerical results}

\section{Implementation details}

\section{Conclusion}

\bibliography{main}

\end{document}
